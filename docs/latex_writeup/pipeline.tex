\documentclass[12pt,preprint]{aastex}

% has to be before amssymb it seems
\usepackage{color,hyperref}
\definecolor{linkcolor}{rgb}{0,0,0.5}
\hypersetup{colorlinks=true,linkcolor=linkcolor,citecolor=linkcolor,
            filecolor=linkcolor,urlcolor=linkcolor}

\usepackage{url}
\usepackage{algorithmic,algorithm}
\usepackage{amssymb,amsmath}

\newcommand{\arxiv}[1]{\href{http://arxiv.org/abs/#1}{arXiv:#1}}

\usepackage{listings}
\definecolor{lbcolor}{rgb}{0.9,0.9,0.9}
\lstset{language=Python,
        basicstyle=\footnotesize\ttfamily,
        showspaces=false,
        showstringspaces=false,
        tabsize=2,
        breaklines=false,
        breakatwhitespace=true,
        identifierstyle=\ttfamily,
        keywordstyle=\bfseries\color[rgb]{0.133,0.545,0.133},
        commentstyle=\color[rgb]{0.133,0.545,0.133},
        stringstyle=\color[rgb]{0.627,0.126,0.941},
    }

\newcommand{\project}[1]{{\sffamily #1}}
\newcommand{\Python}{\project{Python}}
\newcommand{\numpy}{\project{numpy}}
\newcommand{\github}{\project{GitHub}}
\newcommand{\pip}{\project{pip}}
\newcommand{\thisplain}{pipeline}
\newcommand{\this}{\project{\thisplain}}
\newcommand{\paper}{document}
\newcommand{\license}{GNU General Public License v3}

\newcommand{\foreign}[1]{\emph{#1}}
\newcommand{\etal}{\foreign{et\,al.}}
\newcommand{\etc}{\foreign{etc.}}

\newcommand{\Fig}[1]{Figure~\ref{fig:#1}}
\newcommand{\fig}[1]{\Fig{#1}}
\newcommand{\figlabel}[1]{\label{fig:#1}}
\newcommand{\Tab}[1]{Table~\ref{tab:#1}}
\newcommand{\tab}[1]{\Tab{#1}}
\newcommand{\tablabel}[1]{\label{tab:#1}}
\newcommand{\Eq}[1]{Equation~(\ref{eq:#1})}
\newcommand{\eq}[1]{\Eq{#1}}
\newcommand{\eqlabel}[1]{\label{eq:#1}}
\newcommand{\Sect}[1]{Section~\ref{sect:#1}}
\newcommand{\sect}[1]{\Sect{#1}}
\newcommand{\App}[1]{Appendix~\ref{sect:#1}}
\newcommand{\app}[1]{\App{#1}}
\newcommand{\sectlabel}[1]{\label{sect:#1}}
\newcommand{\Algo}[1]{Algorithm~\ref{algo:#1}}
\newcommand{\algo}[1]{\Algo{#1}}
\newcommand{\algolabel}[1]{\label{algo:#1}}

% math symbols
\newcommand{\dd}{\mathrm{d}}
\newcommand{\like}{\mathscr{L}}
\newcommand{\bvec}[1]{\boldsymbol{#1}}
\newcommand{\paramvector}[1]{\bvec{#1}}
\newcommand{\normal}[2]{\mathcal{N} (#1, #2)}
\newcommand{\ensemble}{S}
\newcommand{\colorens}[1]{\ensemble^{(#1)}}
\newcommand{\red}{\colorens{0}}
\newcommand{\blue}{\colorens{1}}
\renewcommand{\vector}[1]{#1}
\renewcommand{\matrix}[1]{#1}
\newcommand{\pr}[1]{\ensuremath{p(#1)}}
\newcommand{\af}{\ensuremath{a_f}}
\newcommand{\expect}[1]{\left<#1\right>}

% model parameters
\newcommand{\model}{\ensuremath{\vector{\Theta}}}
\newcommand{\data}{\ensuremath{\vector{D}}}
\newcommand{\nuisance}{\ensuremath{\vector{\alpha}}}
\newcommand{\link}{\ensuremath{X}}

% units
\newcommand{\unit}[1]{\mathrm{#1}}

% Citation alias
% TODO - \defcitealias{Letey:2018}{LW18}

\begin{document}

\title{\this: TODO}

\newcommand{\cu}{1}
\author{John~Letey\altaffilmark{\cu},
    Tony~E.~Wong\altaffilmark{\cu}}
\altaffiltext{\cu}{University of Colorado at Boulder}

\begin{abstract}
    TODO!!!  The code is available online
    at \url{http://mussles.github.io/\thisplain/} under the \license.
\end{abstract}

% \keywords{
%     methods: data analysis ---
%     methods: numerical ---
%     methods: statistical
% }

~\clearpage

\noindent
\emph{Note: If you want to get started immediately with the \this\ package,
  start at \app{install} on
  page~\pageref{sect:install} or visit the online documentation at
  \url{https://mussles.github.io/pipeline}. If you are sampling with \this\ and having
  low-acceptance-rate or other issues, there is some advice in
  \sect{advice} starting on page~\pageref{sect:advice}.}

\section{Introduction}



\section{The Algorithm}\sectlabel{algo}

\section{Discussion \& Tips}\sectlabel{advice}

\begin{thebibliography}{}\raggedright
\bibitem[Gelman \& Rubin(1992)]{gelman1992}
    Gelman, Andrew; Rubin, Donald B. Inference from Iterative Simulation Using Multiple Sequences. Statist. Sci. 7 (1992), no. 4, 457--472. doi:10.1214/ss/1177011136. \url{https://projecteuclid.org/euclid.ss/1177011136}
\bibitem[Haario et al.(2001)]{haario2001}
    Haario, Heikki; Saksman, Eero; Tamminen, Johanna. An adaptive Metropolis algorithm. Bernoulli 7 (2001), no. 2, 223--242. \url{https://projecteuclid.org/euclid.bj/1080222083}
\bibitem[Rosenthal(2010)]{rosenthal2010}
    Rosenthal, J. S. 2010. Optimal Proposal Distributions and Adaptive MCMC. Handbook of Markov chain Monte Carlo. Eds., Brooks, S., Gelman, A., Jones, G. L., and Meng, X.-L. Chapman \& Hall/CRC Press. Available online at \url{https://pdfs.semanticscholar.org/3576/ee874e983908f9214318abb8ca425316c9ed.pdf}
\bibitem[Roberts(1997)]{roberts1997}
    Roberts, G. O., Gelman, A., and Gilks W. R. 1997. Weak convergence and optimal scaling of random walk Metropolis algorithms. Ann. Appl. Prob. 7, 110–120. Available online at \url{http://projecteuclid.org/download/pdf_1/euclid.aoap/1034625254}
\end{thebibliography}

\clearpage
\appendix
\section{Installation}\sectlabel{install}

\section{Issues \& Contributions}

The development of \this\ is being coordinated on \github\ at
\url{http://github.com/mussles/pipeline} and contributions are welcome. If you
encounter any problems with the code, please report them at
\url{http://github.com/mussles/pipeline/issues} and consider
contributing a patch.

\section{Online Documentation}

To learn more about how to use \this\ in practice, it is best to check out the
documentation on the website \url{https://mussles.github.io/pipeline}. This page includes
the API documentation and many examples of possible work flows.

\end{document}
